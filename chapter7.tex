\chapter{Conclusion}
\label{chap:7}

This chapter concludes this project design report. It gives a brief discussion on the
complete system and measures it against the project objectives set out in Section
\ref{sec:objectives} of this report. Lastly, it discusses possible improvements that can
be made to the system. 

\section{System Performance}

The project objectives set out in Section \ref{sec:objectives} of this report are repeated
here for convenience. They are:

\begin{itemize}
  \item The system must make provision for both NFC and Quick Response Code-based (QR
  Code) payments.
  \item The system must make use of a web server based in the cloud.
  \item An Android application must be made that will allow transactions to be completed
  using NFC.
  \item A demonstration model vending machine must be designed and constructed.
  \item All the data transfers between the user's cellphone and the server must be
  encrypted.
  \item Extra layers of security must be added. 
\end{itemize}

In the final system, both QR Code and NFC technology have been implemented. It was found
that the NFC option has a faster transaction time between product selection and
when the product is dispensed.
However, the NFC option is currently limited to cellphones that have NFC antennas and are
based on the Android Operating System.

It was found that the QR Code payment option typically has a slower transaction time. This
is mainly due to some bottlenecks that the system currently has. These bottlenecks include
the Raspberry Pi's moderate processing power, which increases the encryption time and the
time it takes the Pi to process and decode a live video stream from the webcam. Extra
work can go into optimising the system to make it run faster.

Furthermore, a server has been made that runs in the cloud. This server interacts with a
customer's cellphone through the Android NFC application that was made or the cellphone's
internet browser. All data transactions between the server and the customer's cellphone
are encrypted with an asymmetric encryption scheme with extra layers of security added.

Lastly, a working demonstration vending machine was designed and made. It houses all of
the vending machine components and allows a customer to buy one of two products using
his cellphone. 

\section{Future Work}

The vending machine system that has been designed in this project has been designed to
work as planned. With that being said, there is potential to expand upon the work that has
been completed thus far.

This section discusses some potential improvements and additions that can be made to this
system

\subsection{Commercialisation}

The vending machine currently uses faux money which has no real-world value. Therefore,
the system in its current state is not ready to be deployed across Stellenbosch
University's campus. 

To commercialise the system will require that customers be able to load money
from their bank accounts, credit cards or student accounts onto their vending machine user
accounts. 

This will not be a simple task as this will require integration with the financial
institutions of South Africa. 

\subsection{Polish}

This project focused on the practical and functional aspects of the complete system and
almost no attention was given to the aesthetics of the complete system. Therefore, some
extra work can go into making the web server and Android NFC application more pleasant to use. 

\subsection{Integration}

One important aspect that must be focused on to increase the odds of making this project
more commercially successful, is its integration with current vending machines, i.e. add a
cashless option to current vending machines that are restricted to cash.

This integration may become complicated, as each vending machine manufacturer uses
different components, such as different size motors and control units. 

\subsection{More Payment Options}

Some work can go into expanding the number of payment options available to the user.

This is a feasible option, seeing as the Raspberry Pi has 28 General Purpose Input Output
Pins, of which only 4 are currently being used. These 24 other pins can be used to control
other payment options.

Some options that may be considered are Bluetooth, Short Message System (SMS), Instant
Messaging (such as WhatsApp and BlackBerry Messenger) and Unstructured Supplementary
Service Data (USSD).
