\chapter{Conclusion}
\label{chap:7}

\section{System Performance}

In the final system, both Quick Response Code (QR Code) and Near Field (NFC)
technology, as well as Stellenbosch University (SU) cards, have been
implemented as payment methods. It was found that the NFC and SU cards options
have the shortest transaction time between product selection and when the
product is dispensed.
However, the NFC option is currently limited to cellphones that have NFC
antennas and are based on the Android Operating System and the SU card option
currently has limited user verification functionality and cannot keep track
of a customer's current balance.

It was found that the QR Code payment option typically has a slower transaction time. This
is mainly due to some hardware bottlenecks of the Pi. These bottlenecks include the
Raspberry Pi's moderate processing power, which increases the encryption time and the
time it takes the Pi to process and decode a live video stream from the webcam. These
bottlenecks may be improved with hardware modifications and processor overclocking. Extra
work can also go into optimising the code to make it run more efficiently. This can be
done by converting the code into compiled machine code, such as C or C++.

Furthermore, a server has been made that runs in the computing cloud. This server
interacts with a customer's cellphone through the Android NFC application or a
cellphone's internet browser. All data transactions between the server and the
customer's cellphone are encrypted with an asymmetric encryption scheme with
extra layers of security added.

Lastly, a working demonstration VM was designed and made. It houses all of
the VM components and allows a customer to buy one of two products using
his cellphone. 

\section{Future Work}

The VM system that has been designed in this project has been designed to
specification. With that being said, there is potential to expand upon the work
that has been completed thus far. This section discusses some improvements and additions
that can be made to this system.

\subsection{Commercialisation}

The VM currently uses faux money which has no real-world value. Therefore,
the system in its current state is not ready to be deployed across SU's campus. 

To commercialise the VM, third-party payment providers, such
as PayPal or PayPoint, can be integrated into the system. This will allow customers to
buy credits to use on the VM system by using their credit cards. All the
transactions take place electronically and will therefore be integratable with
the current system.

Furthermore, the memory leakage issue discussed in Sec. \ref{sec:memory_leak} will need
to be addressed and fixed.

\subsection{Polish}

This project focused on the practical and functional aspects of the complete
system and little attention was given to the aesthetics of the complete system.
Therefore, some extra work can go into making the web server and Android NFC
application more user-friendly and improving their interfaces.

\subsection{Integration With Current Systems}

One important aspect that must be focused on to increase the odds of making this project
more commercially successful, is its integration with current VMs, i.e.
adding a cashless option to current VMs that are restricted to cash.



\subsection{More Payment Options}

Some work can go into expanding the number of payment options available to the user.
Some options that may be considered are Bluetooth, Instant Messaging (such as WhatsApp,
BlackBerry Messenger and MXit) and Unstructured Supplementary Service Data (USSD).