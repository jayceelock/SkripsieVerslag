\chapter{Server Configuration}
\label{app:server-config}

\section{Elastic Compute Cloud}

The AWS EC2 server provides the platform on which the Apache
server runs. The EC2 server instance was configured to run
Ubuntu 12.10, `Quantal Quetzal'. This was done because most
of the server development was done on Ubuntu 12.10, and the
server code will therefore require minimal adaptation to be able to run on the EC2
server.

After the server instance was created, the following
packages and programs were installed for the server to be able to run properly:

\begin{itemize}
  \item Apache2: Installs the Apache server framework.
  \item libapache2-mod-wsgi: An Apache module that allows Apache to work with
  Python Web Server Gateway Interface (WSGI) scripts.
  \item python-pip: Allows Ubuntu to install Django from the Python Package Index (PyPI)
  [\cite{website:pypi}].
  \item Django: Installs the all the Django packages that will be used by the server. 
\end{itemize}

Because the server's database uses SQLite3, which is part of the standard Ubuntu
12.10 distribution, no external database programs were needed to be installed.

After this was completed, the server is fully capable of serving Django web
pages.

\section{Apache Configuration}

The Apache server framework provides the foundation on which the Django server runs. It had to
be configured to be able to run the Python scripts that Django contains. To do this, the steps
described in Nick Polet's blog post was followed [\cite{article:apache-setup}]. It describes
in detail how to configure Apache to serve a Django website.

For Apache to be able to serve Django sites, it had to be configured to run the Web Server
Gateway Interface (wsgi.py) script located within the Django server folder. This was done by
adding the following code to the Apache's httpd.conf configuration file:

\begin{verbatim}
WSGIScriptAlias / /home/ubuntu/srv/server_site/server_site/wsgi.py
WSGIPythonPath /home/ubuntu/srv/server_site

<Directory /home/ubuntu/srv/server_site/server_site>
<Files wsgi.py>
Order deny,allow
Allow from all
</Files>
</Directory>
\end{verbatim}

The following line was also needed to be added to Apache's apache2.conf file:

\begin{verbatim}
Include httpd.conf
\end{verbatim}