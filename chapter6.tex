\chapter{System Tests}
\label{chap:6}

In this chapter, the tests  conducted to measure how well the system measures to the
original goals and objectives set out in Section \ref{chap:1}. 

The tests conducted are the system resource usage for the different payment
options and user stories where scenarios are created and the response of the
system is given.

The test results are also discussed. 

\section{System Resource Usage}

A test was conducted to check and compare the system load caused by each of the
different payment methods. The processor load, memory and disk usage were tested
by using the DStat package.

The DStat software ran alongside the software being tested, which also used some
resources. However, this could not be avoided. The resource usage is
present for all three the tests, so it does not unfairly skew the results.

\subsection{Quick Response Code Test}

\subsubsection{Processor Usage}

Figure \ref{fig:qrcode_test_cpu} shows the total processor usage during the complete
Quick Response (QR) Code transaction, as described in Section \ref{sec:transaction} on
page \pageref{sec:transaction}.

It can be observed that there are significant spikes near the maximum capacity when the Pi
starts generating a QR Code and when the return QR Code is scanned. This was expected, as
image and video processing is taxing on hardware and the Pi has no dedicated graphics
processor. The data encryption also adds to the processing load. A possible solution is to
overclock the processor, but it is already overclocked to 900 MHz and pushing the
processor harder may damage the Pi and void its warranty.

It can also be observed that there is a significant time lag between QR Code generation
and the QR Code scanning stages. This is when the user is awaiting authentication from the
server. The only way that this lag can be improved is to move the server location from the
USA to South Africa. It is not expected that doing so will make a significant
difference, however.

\begin{figure}
 \centering 
 \includegraphics[clip=true, trim = 70 260 0 70,
 scale=0.7]{qrcode_test_cpu}
 \caption{QR Code processor usage test results.}
 \label{fig:qrcode_test_cpu}
\end{figure}

\subsubsection{Disk Write Activity}

Figure \ref{fig:qrcode_test_disk} shows the disk writing activity during the QR Code
transaction. It shows several spikes during the transaction. This can possibly be
attributed towards swap file activity by the Pi's operating system and the image encoding
and decoding processes.

Optimising the code may improve this.

\begin{figure}
 \centering 
 \includegraphics[clip=true, trim = 0 530 0 70,
 scale=0.7]{qrcode_test_disk}
 \caption{QR Code disk write test results.}
 \label{fig:qrcode_test_disk}
\end{figure}

\subsubsection{Memory Usage}

Figure \ref{fig:qrcode_test_mem} shows the memory usage during the QR Code transaction
process. It shows a relatively flat graph in the beginning, but goes up toward the end. 

It also shows that the memory used does not return to its start value. This may give rise
to memory leaks during prolonged use. 

The memory performance is discussed in more detail in Section \ref{sec:memory_leak}.

\begin{figure}
 \centering 
 \includegraphics[clip=true, trim = 0 520 0 70,
 scale=0.7]{qrcode_test_mem}
 \caption{QR Code memory usage test results.}
 \label{fig:qrcode_test_mem}
\end{figure}

\subsection{Near Field Communication Test}

\subsubsection{Processor Usage}

Figure \ref{fig:nfc_test_cpu} shows the processor activity during the complete Near
Field Communication (NFC) transaction process. It shows an average load of approximately
50\% during the NFC scanning process.

It can also be observed that the transaction takes under a third of the time it takes for
the QR Code option. 

This is a significant improvement over the QR Code payment option, but code optimisation
and improvements to the Android application may improve this further. 

\begin{figure}
 \centering 
 \includegraphics[clip=true, trim = 0 500 0 70,
 scale=0.7]{nfc_test_cpu}
 \caption{NFC processor usage test results.}
 \label{fig:nfc_test_cpu}
\end{figure}

\subsubsection{Disk Write Activity}

Figure \ref{fig:nfc_test_disk} shows the disk activity during the NFC transaction process.
It shows far less spikes during the transaction than the QR Code option. 

These spikes may again be attributed toward swap file activity from the Pi's operating
system. They are far fewer that the QR Code option, because the NFC payment processing is
less taxing on the processor and uses less memory, as discussed in Section
\ref{sec:nfc_mem}.

\begin{figure}
 \centering 
 \includegraphics[clip=true, trim = 0 530 0 70,
 scale=0.7]{nfc_test_disk}
 \caption{NFC disk write test results.}
 \label{fig:nfc_test_disk}
\end{figure}

\subsubsection{Memory Usage}
\label{sec:nfc_mem}

Figure \ref{fig:nfc_test_mem} shows the memory usage during the NFC payment process. It
shows a relatively flat line, which indicates that the NFC transaction process does is not
memory-intensive. 

This is again a significant improvement over the QR Code option. 

\begin{figure}
 \centering 
 \includegraphics[clip=true, trim = 0 510 0 70,
 scale=0.7]{nfc_test_mem}
 \caption{NFC memory usage test results.}
 \label{fig:nfc_test_mem}
\end{figure}

\subsection{Stellenbosch University Card Test}

\subsubsection{Processor Usage}

Figure \ref{fig:su_test_cpu} shows the processor usage during the Stellenbosch University
(SU) card transaction process. 

It shows only one spike at 35\% while it is scanning for a card. This is far less that the
QR Code option and a slight improvement over the NFC option. 

This increase in performance may be due to the fact that NfcPy (the module used during the
NFC transaction process) is built on top op libnfc and incorporates additional
communication protocols.

The SU transaction process makes use of only libnfc and will therefore most likely be less
processor intensive.

\begin{figure}
 \centering 
 \includegraphics[clip=true, trim = 0 530 0 70,
 scale=0.7]{su_test_cpu}
 \caption{SU Card processor usage test results.}
 \label{fig:su_test_cpu}
\end{figure}

\subsubsection{Disk Write Activity}

Figure \ref{fig:su_test_disk} shows the disk writing activity of the SU transaction
process. 

It looks similar to the NFC transaction disk activity from Figure \ref{fig:nfc_test_disk}.
This may be because they work in a similar manner and both use libnfc to scan for NFC
devices.

\begin{figure}
 \centering 
 \includegraphics[clip=true, trim = 0 550 80 70,
 scale=0.7]{su_test_disk}
 \caption{SU Card disk write test results.}
 \label{fig:su_test_disk}
\end{figure}

\subsubsection{Memory Usage}

Figure \ref{fig:su_test_mem} shows the memory usage during the SU card transaction
process. 

With the exception of the start-up spike, the graph shows a very flat line. This indicates
that the memory usage is very stable and is also relatively low. 

\begin{figure}
 \centering 
 \includegraphics[clip=true, trim = 0 540 80 70,
 scale=0.7]{su_test_mem}
 \caption{SU Card memory usage test results.}
 \label{fig:su_test_mem}
\end{figure}

\subsection{Conclusions}

The resource usage of the QR Code, NFC and SU card transaction processes were analysed. It
was found that the QR Code option is the most memory and process intensive and causes
significant disk activity to take place. 

The NFC and SU card transactions cause similar system loads with regards to memory usage
and disk activity, possibly due to their common code. However, the SU card transaction
process demands approximately 15\% less processing power than the NFC option. 

All three transaction processes can be improved and streamlined, while some hardware
modifications may also decrease the system load. 

\section{User Tests}

The functionality of the system is tested by exposing it to different user scenarios and
seeing what the outcome is. These tests are performed and discussed in this section.

\subsection{User Test 1}

In this user story, the customer decides to pay with a Quick Response Code (QR Code).
He does not yet have a user account on the database and has no credits loaded.
Figure \ref{fig:test1} shows the user story. 

\subsubsection{Scenario Walkthrough}

The user first selects which product to buy from the Raspberry
Pi's User Interface (GUI). The Pi then generates a Quick Response Code (QR Code) that the
user scans with any barcode scanning application. The Universal Resource Locator (URL)
embedded in the QR Code then takes the customer to a web page located on the server. 

Not having a user account, the user clicks on the lick to create a new user profile. After
entering valid user information, the customer is returned to the login  page where he
can log in with his new login credentials. 

After logging in, the server displays to the customer that he does not yet have any
credits loaded onto his account. The customer then goes to the load credit page where
they load some money. When this is done, the server displays a confirmation QR Code on
the customer's cellphone screen, verifying the transaction.

The customer then presses the continue button on the GUI, which starts scanning the web
cam feed for the customer's QR Code. After this is done, the Pi activates the correct
motor and dispenses the product.

\subsubsection{Results}

In this scenario, a customer successfully loaded credits onto his account and bought a
product with the new credits. There are many steps involved in the process, however, and
this leads to a relatively long transaction process. 

It was also found that the server is sometimes unable to correctly decrypt the code it
receives from the customer's cellphone. It is still unclear why this happens, but it is
possible that the barcode scanning application on the cellphone incorrectly decodes the
vending machine's QR Code. 

This issue will have to be fixed before the system is commercialised.

\begin{figure}
 \centering 
 \includegraphics[clip=true, trim = 0 410 0 60,
 scale=0.7]{user_story_1}
 \caption{The first user scenario.}
 \label{fig:test1}
\end{figure}

\subsection{User Test 2}

In this scenario, a customer tries to use a QR Code from a previous transaction. Figure
\ref{fig:test2} shows the user story.

\subsubsection{Scenario Walkthrough}

To begin, the customer selects a product to buy, after which the Raspberry Pi displays a
QR Code. However, instead of scanning the QR Code and paying for a new product, the
customer tries to use an old QR Code that the server sent him after a successful
transaction previously. 

The customer tells the Raspberry Pi to scan his old QR Code. The Raspberry Pi does
this, but the old QR Code does not contain a valid response to the Raspberry Pi's
challenge. Therefore, the transaction is valid and the customer is informed of his
invalid QR Code.

\subsubsection{Results}

The results from this scenario shows that the challenge-response system works as it was
designed to: it does not allow an old QR Code to be used again without it being paid for. 

\begin{figure}
 \centering 
 \includegraphics[clip=true, trim = 0 550 0 70,
 scale=0.7]{user_story_2}
 \caption{The second user scenario.}
 \label{fig:test2}
\end{figure}

\subsection{User Test 3}

In this scenario, the user opts to buy a product using the Android Near Field
Communication (NFC) application. However, a new user profile must first be created. The user story
can be seen in Figure \ref{fig:test3}.

\subsubsection{Scenario Walkthrough}

After the user opens the application, he is presented with a login screen. The
user enters his login credentials and checks the box that confirms that he is a
new user. The application then contacts the server with a user creation
request. Provided that the username is not already in use, the server sends the
application a confirmation.

The application then returns the user to the main menu where the user can pick
a product to buy. After picking a product, the application contacts the server
with a purchase request. If the transaction is successful, the server sends the
application a confirmation code which the application transmits via the
cellphone's NFC antenna.

The user then swipes his phone across the vending machines NFC receiver. The
vending machine then activates the motor and the user receives his product. 

\subsubsection{Results}

In this scenario, a customer has successfully used the Android NFC payment application to
create a new user profile and buy a product from the vending machine. 

The process is more streamlined than the QR Code option and consists of fewer steps. This
makes the NFC transaction process take place in a shorter period of time. After the
customer has opened the application once, he will not have to enter his login details
again. This will make the process slightly quicker in the future.  

\begin{figure}
 \centering 
 \includegraphics[clip=true, trim = 0 450 0 70,
 scale=0.7]{user_story_3}
 \caption{The third user scenario.}
 \label{fig:test3}
\end{figure}