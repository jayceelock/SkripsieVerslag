\chapter{System Tests}
\label{chap:6}

In this chapter, the tests  conducted to measure how well the system measures to the
original goals and objectives set out in Section \ref{chap:1}. 

The tests conducted are the system resource usage for the different payment
options and user stories where scenarios are created and the response of the
system is given.

The test results are also discussed. 

\section{System Resource Usage}

A test was conducted to check and compare the system load caused by each of the
different payment methods. The processor load, memory and disk usage were tested
by using the DStat package.

The DStat software ran alongside the software being tested, which also used some
resources. However, this could not be avoided. The resource usage is
present for all three the tests, so it does not unfairly skew the results.

\subsection{Quick Response Code Test}

\subsection{Near Field Communication Test}

\subsection{Stellenbosch University Card Test}

\section{User Tests}

The functionality of the system is tested by exposing it to different user scenarios and
seeing what the outcome is. These tests are performed and discussed in this section.

\subsection{Test 1}

In this user story, the customer decides to pay with a Quick Response Code (QR Code).
He does not yet have a user account on the database and has no credits loaded.
Figure \ref{fig:test1} shows the user story. 

The process is as follows: The user first selects which product to buy from the Raspberry
Pi's User Interface (GUI). The Pi then generates a Quick Response Code (QR Code) that the
user scans with any barcode scanning application. The Universal Resource Locator (URL)
embedded in the QR Code then takes the customer to a web page located on the server. 

Not having a user account, the user clicks on the lick to create a new user profile. After
entering valid user information, the customer is returned to the login  page where he
can log in with his new login credentials. 

After logging in, the server displays to the customer that he does not yet have any
credits loaded onto his account. The customer then goes to the load credit page where
they load some money. When this is done, the server displays a confirmation QR Code on
the customer's cellphone screen, verifying the transaction.

The customer then presses the continue button on the GUI, which starts scanning the web
cam feed for the customer's QR Code. After this is done, the Pi activates the correct
motor and dispenses the product.

\begin{figure}
 \centering 
 \includegraphics[clip=true, trim = 0 470 0 50,
 scale=0.7]{user_story_1}
 \caption{The first user scenario.}
 \label{fig:test1}
\end{figure}

\subsection{Test 2}

In this scenario, a customer tries to use a QR Code from a previous transaction. Figure
\ref{fig:test2} shows the user story.

To begin, the customer selects a product to buy, after which the Raspberry Pi displays a
QR Code. However, instead of scanning the QR Code and paying for a new product, the
customer tries to use an old QR Code that the server sent him after a successful
transaction previously. 

The customer tells the Raspberry Pi to scan his old QR Code. The Raspberry Pi does
this, but the old QR Code does not contain a valid response to the Raspberry Pi's
challenge. Therefore, the transaction is valid and the customer is informed of his
invalid QR Code.

\begin{figure}
 \centering 
 \includegraphics[clip=true, trim = 0 620 0 50,
 scale=0.7]{user_story_2}
 \caption{The second user scenario.}
 \label{fig:test2}
\end{figure}

\subsection{Test 3}

In this scenario, the user opts to buy a product using the Android Near Field
Communication (NFC) application. However, a new user profile must first be created. The user story
can be seen in Figure \ref{fig:test3}.

After the user opens the application, he is presented with a login screen. The
user enters his login credentials and checks the box that confirms that he is a
new user. The application then contacts the server with a user creation
request. Provided that the username is not already in use, the server sends the
application a confirmation.

The application then returns the user to the main menu where the user can pick
a product to buy. After picking a product, the application contacts the server
with a purchase request. If the transaction is successful, the server sends the
application a confirmation code which the application transmits via the
cellphone's NFC antenna.

The user then swipes his phone across the vending machines NFC receiver. The
vending machine then activates the motor and the user receives his product. 

\begin{figure}
 \centering 
 \includegraphics[clip=true, trim = 0 500 0 50,
 scale=0.7]{user_story_3}
 \caption{The third user scenario.}
 \label{fig:test3}
\end{figure}