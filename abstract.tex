\begin{abstract}
   
   Currently, Stellenbosch University only has cash-based vending machines available. This
   means that a physical cash transaction has to take place before the vending machine can
   dispense its products. In a world moving away from using cash, however, this payment
   approach becomes a problem as less people are likely to carry around cash on their
   person. Therefore it is important that vendors and manufacturers keep up
   with the trend and implement the latest technologies in their products and
   payment methodologies.
   
   This project focuses on introducing a vending machine system that will allow customers
   to pay for their products using only their cell phones. The project was designed
   to use existing technologies, services and protocols wherever possible. The system is
   based mainly on the Python programming language.
   
   The two main technologies implemented are Quick Response Codes (QR Codes) and Near
   Field Communication (NFC). An Android application was created to facilitate payments
   made via NFC. A cloud-based server was also created and is used by the vending machine
   and Android application to validate transactions. To prove that the system
   works, a model vending machine was designed and built that uses the cashless
   payment system.
   
   The results of this project are favourable and show that it is possible to make a
   cashless vending machine. However, more work still needs to be done to fully
   commercialise this system and make it user friendly. 
   
\end{abstract}

\endinput