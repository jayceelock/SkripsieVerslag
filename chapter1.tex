\chapter{Introduction}
\label{chap:1}

\section{Problem Statement}

The vending machines currently used at Stellenbosch University (SU) exclusively
make use of cash transactions. These vending machine systems are currently the
de facto standard throughout the world, but they do have one drawback:
they require a hard cash transaction to take place. In a world moving away from
cash transactions and towards online payments, e-transactions and mobile
payments, this may become a problem to potential customers.

With that in mind, a need has been identified at SU for a vending machine that accepts
cashless transactions.

\section{Existing Solutions}

Currently there are cashless payment solutions being used by the general public.
These include debit and credit cards, Radio Field Identification (RFID) cards,
Unstructured Supplementary Service Data (USSD) based systems and, more recently, Near
Field Communication (NFC) payments.

These aspects will be further discussed in this section.

\subsection{Credit and Debit Cards}

A well-known and widely-used alternative to cash payments are the debit and
credit cards most modern adults carry on their person. This is especially
true in developed countries with mature and reliable financial institutions.

The great advantages these cards hold over the other cashless solutions are
that they are relatively easy to obtain from a bank and that they have become very
reliable and simple to use.

A possible disadvantage is that such systems may become costly and complicated to implement
because of each banks different security and transfer protocols. This is especially true
for a simple system such as the one developed for this project.

\subsection{Radio Field Identification}

Radio Field Identification (RFID) is a technology which was first patented in 1983
[\cite{patent:nfc-patent}]. Since then, the technology has grown and matured into a very
reliable identification and payment platform. 

Examples of where this is used for making payments is the payments made to new parking meters
with a contactless card. 

The advantage of this technology is its great convenience: a customer only needs
to tap the card against a receiver and it is not required that a password be
entered.

However, this leads to some security concerns, for example if the card gets stolen
or cloned, the thief can use the money on the card for her own benefit.
Fortunately, these cards most commonly work with pre-paid money.
Therefore, provided that there was not too much money loaded onto the card, the
theft victim will not suffer a big financial loss. 

Therefore, an RFID card has the same risk as regular cash.

\subsection{Unstructured Supplementary Service Data}

Unstructured Supplementary Service Data (USSD) is a communication standard used by cellphones
to exchange data with a service provider's servers. If the service provider allows it, USSD may
be used by a customer to make financial transfers. 

An example of this is the M-Pesa mobile money service in Kenya, which is based on the use of
USSD. It allows for a customer to pay for goods ranging from milk to bread and even the
monthly rent. It is currently regarded as the most advanced and popular mobile payment
platform in the world [\cite{journal:m-pesa}]. 

An advantage of implementing such a system is that it has been proven to be reliable and
is usable by almost any cellphone.

The disadvantage is that it requires third party vendors, such as the mobile service
providers, to provide systems and services. This may add unnecessary costs
to the system.

\subsection{Near Field Communication}

Near Field Communication (NFC) payments have recently come to the fore as a
prominent method of making cashless transactions. Especially in Europe and North
America, there have been significant advances in making this payment method a
more attractive solution. Google has been making the largest contribution with
the addition of NFC protocols to its Android
platform [\cite{website:android-gingerbread}].

Some examples of NFC based payments are the London public transport system, which makes
provision for NFC payments [\cite{article:nfc-underground}], as well as some retail vendors
which accept payments made via Google's Wallet application.

\section{Goal of the Final System}
\label{sec:final-system-goal}

Although hard cash still remains the largest contributor to global financial transactions,
standing at 59\% of the 37 billion transactions that took place in 2012
[\cite{article:money-transactions}, \cite{website:money-transactions}]. However,
mobile and card transactions are expected to surpass cash as the leading
payment method by 2015 [\cite{article:cashless-transactions}].

To this end, mobile payments, i.e. payments made with a cellphone, was chosen
as the medium to facilitate cashless payments for this vending machine.
Therefore, the final goal of this project would be to deliver a vending machine
that will be used on SU's campus and will allow anyone to buy products from
the vending machine using only their cellphones.

\section{System Objectives}
\label{sec:objectives}

The system objectives are:

\begin{itemize}
  \item The system must make provision for both NFC and Quick Response Code-based (QR
  Code) payments.
  \item The system must make use of a web server based in the cloud, i.e. it
  must be accessible by anyone across the world.
  \item An Android application must be made that will allow transactions to be completed
  using NFC.
  \item A demonstration model vending machine must be designed and constructed.
  \item All the data transfers between the user's cellphone and the server must be
  encrypted.
  \item Extra layers of security must be added to prevent theft and product
  losses.
\end{itemize}

\section{Report Structure}

In this report, Chapter \ref{chap:2} will give background information on all the
technology, concepts and programs used in this project. Thereafter in Chapter
\ref{chap:3}, the overall system design is discussed, which will be followed by
a discussion on the detailed design of the software and hardware aspects of the
entire system in Chapters \ref{chap:4} and \ref{chap:5}. Afterwards in
Chapter \ref{chap:6}, the system test results will be discussed and analysed,
which will be followed by a discussion on the complete system and finally the
conclusion on the project in Chapter \ref{chap:7}.
