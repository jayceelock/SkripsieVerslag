\chapter{Introduction}

\section{Problem Statement}

The vending machines currently used at the Stellenbosch University (SU) make exclusive use of
cash transactions. These vending machine systems are the de facto standard throughout the
world and have been for the largest part of the last two centuries, but they do have one
drawback: they require a hard cash transaction to take place and in a world moving away from
cash transactions and toward online payments, e-transactions and mobile payments, this may
become a problem to potential customers.

With that in mind, a need that has been identified is for a vending machine that accepts
cashless transactions.

\section{Existing Solutions}

Currently there are plenty of cashless payment options being used by the general public. These
include debit and credit cards, Radio Field Identification (RFID) cards, Unstructured
Supplementary Service Data (USSD) based systems and, more recently, Near Field Communication.

\subsection{Credit and Debit Cards}

An well-known and convenient alternative to cashless solutions are the plastic cards most
modern adults carry around. This is especially true in first world countries with mature
banking systems. 

The great advantage these cards holds over the alternatives are that they are
relatively easy to get from a bank and that they have become very reliable. 

A possible disadvantage is that such systems may become costly and complicated to implement
because of each banks different security and transfer protocols.

\subsection{Radio Field Identification}

Radio Field Identification (RFID) is a technology which was first patented in 1983
[\cite{patent:nfc-patent}]. Since then the technology has grown and matured into a very
reliable identification and payment platform. 

Examples of where this is used for making payments is the payments made to new parking meters
with a contactless card. 

The advantage of this technology is its great convenience: a customer only needs to swipe and
it it not required that a password be entered.

However, this leads to some security concerns because if the card gets stolen, the thief can
use the money on the card for his/her own gain. Thankfully though, these cards most commonly
work with pre-loaded money, so provided that there wasn't too much money loaded onto the card,
the theft victim will not be too badly off. 

\subsection{Unstructured Supplementary Service Data}

Unstructured Supplementary Service Data (USSD) is a communication standard used by cellphones
to exchange data with a service provider's servers. If the service provider allows it, USSD may
be used by a customer to transfer money from one account to another. 

An example of this is the M-Pesa mobile money service in Kenya, which is based on the use of
USSD. It allows for a customer to pay for goods ranging from milk, bread, even the monthly
rent. Its currently regarded as the most advanced mobile payment platform in the world
[\cite{website:m-pesa}]. 

An advantage of implementing such a system is that it is proven to work and is usable by almost
any cellphone.

The disadvantage is that it requires third party vendors to provide systems and services, which
will add unnecessary costs to the system.

\subsection{Near Field Communication}

Near Field Communication (NFC) payments have recently come to the fore as a likely candidate to
become the standard method of mobile payment. Especially in Europe and North America, there
have been significant advances in making this payment method a more attractive solution, with
Google making the largest contribution with the addition of NFC protocols to its Android
platform.

Some examples of NFC based payments are the London public transport system, which makes
provision for NFC payments [\cite{website:nfc-underground}], as well as some retail vendors
which accept payments made via Google's Wallet application.

\section{Goal of Final System}

Although hard cash still remains the largest contributor to global financial transactions,
standing at 59\% of the 37 billion transactions that took place in 2012 [\cite{website:money-transactions}], however,
mobile and card transactions are expected to surpass cash as the leading payment method by 2015
[\cite{website:money-transactions}]. 

The main goal of this project is to develop a vending machine that will make use of this
increase in cashless payment by adding the option to pay for a product with a customer's
cellphone. 

\section{System Objectives}

The system objectives are:

\begin{itemize}
  \item Must have at least two methods of mobile payment.
  \item The solution must be built while keeping commercialisation in mind.
  \item 
\end{itemize}

\section{Report Structure}

In this report, some background information on all the technology, concepts and programs used
in this project is given. Then the overall system design is discussed, which is followed by
a discussion on the detailed design of the whole system. Then the system tests are discussed
and analysed, which is then followed by a discussion on the complete system and a
project conclusion.
