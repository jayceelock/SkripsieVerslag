\chapter{Techno-Economic Report}
\label{app:techno-economic}

\section{Budget}

The planned total cost of completing this project was R532 000, with R12 000 going toward
manufacturing and components and R520 000 going toward the designer's salary.

The final cost for the design and manufacture of the complete system is R534 596. The cost
breakdown for this can be seen in Table \ref{tab:cost-breakdown}. The actual system value
is R14 596.

\begin{table}
\caption{Cost breakdown of the project and system}
\label{tab:cost-breakdown}
\centering
\begin{tabular}{|l|l|l|l|}
  \hline
  \textbf{Component} & \textbf{Amount required} & \textbf{Bought?} & \textbf{Cost when
  new} \\\hline\hline 
  12DC Motor & 2 & No & R4000 \\\hline
  NFC Controller & 1 & Yes & R780 \\\hline
  PS2 Eye Toy Webcam & 1 & No & R200 \\\hline
  12V Relay & 2 & No & R20 \\\hline
  2N2222 BJT & 2 & No & R5 \\\hline
  Vending machine unit & 1 & Yes & R107 \\\hline
  Raspberry Pi & 1 & No & R400 \\\hline
  HDMI Monitor & 1 & No & R5000 \\\hline
  Coils & 2 & No & R30 \\\hline
  Web server & 1 & No & Free \\\hline
\end{tabular}
\end{table}

Not all of the components used in this project were bought new. Some were loaned from
Stellenbosch University or elsewhere. Therefore, the actual cost of completing the project
is R520 887. 

Possible areas where saving can be made is to use smaller, cheaper motors and a smaller,
less expensive computer monitor. A smaller monitor will cost approximately R1000 and if
the system is integrated with a cash-based vending machine, the cost for the DC motors
will disappear. This gives a total system cost of R 3 596.

\section{Time Management}

This project was planned to be completed within 10 months. It was completed ahead of
schedule. A Gannt chart showing the project timeline can be seen in Figure \ref{fig:gannt}
in Appendix \ref{app:c}.

\section{Technical Impact}

This project produced a vending machine which accepts payments made via cell phone.
Allowing customers to pay using only their cell phones makes an already convinient service
even more convenient. 

The prototype system was designed with expansion and modification in mind an is not
necessarily restricted to vending machines.
For example, the system will accept any cell phone payment made at any vendor if it is
properly modified and configured. 

\section{Return on Investment}

\textbf{[Prof. van Rooyen: Dit voel vir my die potential for commercialisation is klaar in
hierdie afdeling behandel. Moet ek dit opbreek in twee dele liewers, of kan ek dit so
hou?]}

As discussed in Section \ref{sec:final-system-goal} of the main report, cashless
transactions are expected to surpass cash transactions by the year 2015. It is therefore
very important for vendors to keep up with the trend and allow customers to pay with
cash, but still provide a cashless payment facility. 

This project has delivered a working cashless vending machine, but the system can be
expanded upon and adapted to be integrated with traditional, cash-based vending
machines. This integration will require more research and development, however, but it
will be worth it to give modern shoppers a choice between paying with or without cash.

To get to this point, an estimated R2.3million will be required. R1.1million will go
toward paying the developers' salaries, R1million toward setting up data servers and paying their
overhead costs, R100 000 for components and manufacturing costs and a final R100 000 for
buying at least three standard cash-based vending machines to test the final system. 

When the final system is complete, it presents a few opportunities for investors to make
money. Firstly, the systems can be manufactured by the development company and sold to
vending machine manufacturers. Second, it can be manufactured under license by the vending machine
manufacturers, with the development company receiving royalties. Lastly, the system can be
patented and the patent can be sold to another company for a premium. 

If the system is manufactured in-house, it will cost approximately R3 596 to make. If
these systems are then sold to vending machine manufacturers at R7000 per unit and 210
units are sold per year, the original investment will be remade approximately 4 years,
not accounting for inflation. This gives a return of 18.9\% per year.

If these systems are manufactured under license by vending machine companies, and
royalties are received in the amount of R2300 per unit made with 210 units made per year,
it will take 6 years for the investors to remake their money. This gives a rate of return
of 12.2\% per year. 

Lastly, if the patent is sold, it will have to be sold for at least R2.9million to make up
for the initial investment. It is highly unlikely that a company will spend this amount of
capital on an unproven system. If it is sold for R3million, it will immediately give
investors a return on their investment. However, the long-term revenue generated by the
other two options far outweigh this R3million and it is therefore recommended that the
first two options be considered.
